%%%%%%%%%%%%%%%%%%%%%%%%%%%%%%%%%%%%%%%%%
% Developer CV
% LaTeX Class
% Version 2.0 (modified) (12/09/23)
%
% This class originates from:
% http://www.LaTeXTemplates.com
%
% Author:
% Donovan Clay
% Based on a template by Omar Roldan
% Based on a template by Jan Vorisek (jan@vorisek.me)
% Based on a template by Jan Küster (info@jankuester.com)
% Modified for LaTeX Templates by Vel (vel@LaTeXTemplates.com)
%
% License:
% The MIT License (see included LICENSE file)
%
%%%%%%%%%%%%%%%%%%%%%%%%%%%%%%%%%%%%%%%%%

%----------------------------------------------------------------------------------------
%	PREAMBLE: PACKAGES AND OTHER DOCUMENT CONFIGURATIONS
%----------------------------------------------------------------------------------------


% !TEX program = xelatex
% https://stackoverflow.com/questions/56109128/enable-xelatex-in-latex-workshops-for-visual-studio-code

\documentclass[10pt]{developercv} % Default font size, values from 8-12pt are recommended
\usepackage{multicol}
\usepackage{tabularx}
\usepackage{ragged2e}
\setlength{\columnsep}{0mm}
%----------------------------------------------------------------------------------------
\usepackage{lipsum}  

\newcommand{\firstname}{Donovan}
\newcommand{\lastname}{Clay}
\newcommand{\name}{\firstname\ \lastname}
\newcommand{\currentjob}{Software Engineer}
% \newcommand{\homepage}{\href{https://dorianclay.com}{dorianclay.com}}
\newcommand{\email}{\href{mailto:donovan@donovanclay.com}{donovan@donovanclay.com}}
\newcommand{\phone}{425.974.9100}
\newcommand{\location}{}
\newcommand{\github}{\href{https://github.com/donovanclay}{github.com/donovanclay}}
\newcommand{\linkedin}{\href{https://www.linkedin.com/in/donovanclay}{/in/donovanclay}}
\newenvironment{condenseditemize}{\begin{itemize}[noitemsep, topsep=0ex, parsep=0ex, partopsep=0ex, leftmargin=1em]}{\end{itemize}}


\begin{document}

\begin{minipage}[t]{0.5\textwidth} 
	\vspace{-\baselineskip} % Required for vertically aligning minipages
	
	{ \fontsize{16}{20} \textcolor{black}{\textbf{\MakeUppercase{\name}}}} % First name
	
	\vspace{6pt}
	
	{\Large \currentjob} % Career or current job title
\end{minipage} % must not have white space between minipages
\hfill
\begin{minipage}[t]{0.2\textwidth} % 20% of the page width for the first row of icons
	\vspace{-\baselineskip} % Required for vertically aligning minipages
	
	% The first parameter is the FontAwesome icon name, the second is the box size and the third is the text
	% \icon{Globe}{9}{\homepage}\\ 
    \icon{Phone}{9}{\phone\hfill}\\
    % \icon{MapMarker}{11}{\location}\\
	
\end{minipage}
\begin{minipage}[t]{0.27\textwidth} % 27% of the page width for the second row of icons
	\vspace{-\baselineskip} % Required for vertically aligning minipages
	
	\icon{Envelope}{9}{\email}\\	
    \icon{Github}{9}{\github}\\
    % \icon{LinkedinSquare}{11}{\linkedin}\\    
\end{minipage}
% %----------------------------------------------------------------------------------------
%	INTRODUCTION, SKILLS AND TECHNOLOGIES
%----------------------------------------------------------------------------------------

\begin{minipage}[t]{0.46\textwidth}
    \cvsect{Summary}
	\vspace{-6pt}
 
    Interdisciplinary software engineer striving for an autonomous, private, and accessible Internet.
\end{minipage}
\hfill % Whitespace between
\begin{minipage}[t]{0.465\textwidth}
    \cvsect{Skills}
    \vspace{-6pt}
    
    \begin{minipage}[t]{0.2\textwidth}
        \textbf{Languages:}
    \end{minipage}
    \hfill
    \begin{minipage}[t]{0.73\textwidth}
        C\#, Python, JavaScript, C++, HTML.
    \end{minipage}
    \vspace{4mm}
    
    \begin{minipage}[t]{0.2\textwidth}
        \textbf{Technologies:}
    \end{minipage}
    \hfill
    \begin{minipage}[t]{0.73\textwidth}
        Azure, .NET, Browser, Jupyter, Node.JS.
    \end{minipage}
    
\end{minipage}

%----------------------------------------------------------------------------------------
%	WORK EXPERIENCE
%----------------------------------------------------------------------------------------
\vspace{\sectionpadding}
\vspace{-0.5ex}
\cvsect{Work Experience}
\vspace{-1.5ex}
\entry
    {Amazon (AWS SageMaker, Seattle)}
    {Software Development Intern}
    {Jun. 2024 - Sep. 2024}
    {
    Setup CICCD infrastructure to automate the testing and release process of the SageMaker CodeEditor app.
    \begin{condenseditemize}
        \item Eliminated 2-3 weeks of work per release of new version.
        \item Used AWS Cloud Development Kit to write ``infrastructure as code''.
        \item Triggered 
    \end{condenseditemize}
    }

\entry
    {University of Washington, Paul G. Allen School of CSE}
    {Part-time Teaching Assistant}
    {Jan. 2024 - Mar. 2024}
    {
    I assisted in teaching a Digital Circuit Design class of 43 students.
    \begin{condenseditemize}
        \item Hosted support hours 5 hours a week to help students understand concepts from class and debug their projects
        \item Coordinated with a teaching staff of 5 other people to grade weekly assignments
    \end{condenseditemize}
    }

% \entry
%     {Eternium Associates}
%     {Custom Home Remodeling}
%     {Jun. 2022 - Sep. 2022}
%     {
%     \begin{condenseditemize}
%         \item Frequently used problem solving skills to produce original, cost-effective, efficient, and high-quality solutions.
%         \item with Agile-based management, adhered to a consistent schedule, to achieve deadlines and meet client expectations.
%         \item Communicated with management and 2-3 coworkers to coordinate work.
%     \end{condenseditemize}}
%----------------------------------------------------------------------------------------
%	RESEARCH EXPERIENCE
%----------------------------------------------------------------------------------------
\vspace{\sectionpadding}
\cvsect{Research Experience}
\entry
    {\href{https://www.routledge.com/Exploring-Extended-Realities-Metaphysical-Psychological-and-Ethical-Challenges/Kissel-Ramirez/p/book/9781032417325}{UW ACME Lab}}
    {HLS4ML Team}
    {Jan. 2024 - Current}
    {
    Team is developing software to convert from neural network code, written in Python, to SystemVerilog, which implements the ML models in hardware on FPGAs.
    \begin{condenseditemize}
        \item Developed benchmarks to test handmade SystemVerilog models against the SystemVerilog model generated by the HLS4ML tool
        \item Coordinated with a team from Drexel University to collaborate on a system to monitor material deposition
        \item Investigated errors in the HLS4ML implementation of the Keras BatchNormalization layer
    \end{condenseditemize}
    }

% %----------------------------------------------------------------------------------------
%	WORKS IN PROGRESS
%----------------------------------------------------------------------------------------
\vspace{\sectionpadding}
\cvsect{Works in Progress}
\entry
    {Embodiment, Relationships, and Sexuality in the Metaverse: A Critical Analysis}
    {}
    {}
    {A paper exploring how extended-reality embodiment and virtually real experiences challenge normative concepts like relationships, sex, and sexual orientation. Paper also considers moral deskilling and proper target concerns raised when relationships are mediated by extended reality and AI.
    }

% \pagebreak
% %----------------------------------------------------------------------------------------
%	PUBLICATIONS
%----------------------------------------------------------------------------------------
\vspace{\sectionpadding}
\cvsect{Publications}
\entry
    {\href{https://www.routledge.com/Exploring-Extended-Realities-Metaphysical-Psychological-and-Ethical-Challenges/Kissel-Ramirez/p/book/9781032417325}{Extended reality, control, and problems of the self: An ethical analysis}}
    {}
    {Dec 2023}
    {In \textit{Exploring Extended Realities: Metaphysical, Psychological, and Ethical Challenges}}

\entry
    {\href{https://kdd.org/kdd2022/papers/17_Dorian\%20Clay.pdf}{Expanding Neuro-Symbolic Artificial Intelligence for Strategic Learning}}
    {}
    {Aug 2022}
    {SIGKDD 2022 Undergraduate Consortium}

\entry
    {\href{https://scholarcommons.scu.edu/cseng_senior/228}{Expanding Neuro-Symbolic Artificial Intelligence for Strategic Learning}}
    {Undergraduate Thesis}
    {Jun 2022}
    {Santa Clara University 52nd Annual School of Engineering Senior Design Conference}

% %----------------------------------------------------------------------------------------
%	PRESENTATIONS
%----------------------------------------------------------------------------------------
\vspace{\sectionpadding}
\cvsect{Conference Presentations}
\entry
    {\href{https://www.scu.edu/engineering/undergraduate/senior-design/archives/2022-senior-design/}{Expanding Neuro-Symbolic Artificial Intelligence for Strategic Learning (feat. Blokboi)}}
    {}
    {May 2022}
    {Santa Clara University 52nd Annual School of Engineering Senior Design Conference}


\entry
    {The Metaverse and Problems of the Self}
    {}
    {Apr 2022}
    {Great Lakes Philosophy Conference}

%----------------------------------------------------------------------------------------
%	Projects
%----------------------------------------------------------------------------------------
\vspace{\sectionpadding}
\cvsect{Projects}
\entry
    {Indoor Air Quality Controller}
    {\quad \href{https://github.com/donovanclay/isy}{https://github.com/donovanclay/isy}}
    {Jul. 2022 - Sep. 2023}
    {
        This controller manages the supply and exhaust fans in a house to keep stable air pressure and humidity.
        \begin{condenseditemize}
            \item Communicated with APIs to poll local weather, air quality, and the house's security system

            \item Used sensors to detect airflow, humidity, motion, and occupancy and control fans based on that data
            
            \item Used Python sockets to implement a Websocket to send live updates about the state of the application
            
            \item Used React and NextJS to develop and Docker to deploy a local website to show real time diagnostic data
            
            \item Used Python to implement logic to equalize exhaust airflow with supply airflow
        \end{condenseditemize}
    }

% \entry
%     {\href{https://github.com/donovanclay/foodbot}{``Foodbot'' Discord Bot}}
%     {\quad \href{https://github.com/donovanclay/foodbot}{https://github.com/donovanclay/foodbot}}
%     {Jan. 2023 - May 2023}
%     {
%         This is a general purpose Discord bot with many fun, random features. It can check when UW CS homework is released, generate images, chat with the user with different personalities, and track keyword occurrences and typos.
        
%         APIs used: ChatGPT, Bing Spell Check, Stability AI Image Generation.
%         \begin{condenseditemize}
%             \item Harnesses APIs for features like typo recognition, prompt based image generation, and a chat bot with configurable personalities

%             \item Uses REST to poll UW CSE course websites to download homework
%         \end{condenseditemize}
%     }

\entry
    {\href{https://devpost.com/software/illustraitor}{illustrAItor}}
    {Finalist in UW Hackathon 2022 \quad\href{https://github.com/donovanclay/illustrAItor}{https://github.com/donovanclay/illustrAItor}}
    {Oct. 2022}
    {
        This project makes reading more enjoyable by complementing it with customizable AI-generated illustrations. illustrAItor creates captivating illustrations to complement the passage of text being read.\\
        APIs used: Stability AI Image Generation, Google Books.
        \begin{condenseditemize}
            \item Produced the project in 24 hours in a highly competitive environment
            \item Worked in a small group of 4 for long hours
            \item Used multiple API's to pull and serve information to and from different services
        \end{condenseditemize}
    }
    
\enlargethispage{10\baselineskip}
% %----------------------------------------------------------------------------------------
%	AWARDS
%----------------------------------------------------------------------------------------
\vspace{\sectionpadding}
\cvsect{Awards}
\entry
    {\href{https://www.scu.edu/ethics/focus-areas/campus-ethics/programs-for-students/co-curricular-activities/ethics-prizes/}{Markkula Center Engineering Ethics Prize}}
    {Santa Clara University}
    {Jun 2022}
    {This prize is awarded for best analysis of ethical issues related to the senior thesis. My paper assessed whether my hybrid model improved interpretability, finding it didn’t. I argued the neural part of the model stayed a black box, although reasoning via symbolic logic was more interpretable.}

\entry 
    {Santa Clara University Dean's List}
    {}
    {Jun 2019}
    {Top 10\% of School of Engineering students by GPA.}
%----------------------------------------------------------------------------------------
%	EDUCATION
%----------------------------------------------------------------------------------------
\vspace{\sectionpadding}
\cvsect{Education}
% \vspace{-0.5ex}
\entry
    {\href{https://www.cs.washington.edu/academics/ugrad}{B.S. Computer Engineering with Honors}}
    {University of Washington}
    {Sep. 2022 - Jun. 2025}
    {
        Paul G. Allen School of Computer Science and Engineering \hfill \color{gray} 3.92 Major GPA \hspace{-0.62em}
    }
\vspace{-0.5ex}
\entry 
    {\href{https://chid.washington.edu/undergraduate-programs}{B.A. Comparative History of Ideas}}
    {University of Washington}
    {Sep. 2022 - Jun. 2025}
    {
        Interdisciplanary degree in social sciences, philosophy, communications, sociology, and more \hfill \color{gray} 3.92 Major GPA \hspace{-0.62em}
    }
\vspace{-0.5ex}
\entry 
    {\href{https://www.bellevuecollege.edu/}{Computer Engineering Prerequisites}}
    {Bellevue College}
    {Sep. 2020 - Jun. 2022}
    {
        Transfered to University of Washington \hfill \color{gray} 3.93 GPA \hspace{-0.62em}
    }
\vspace{\sectionpadding}
\cvsect{Skills}
\setlength{\hsize}{0.9\hsize}% emphasize effects
% \begin{justify}
\justifying\text{Python - Java - C/C$++$ - Rust - AWS - SystemVerilog - FPGAs - PyTorch - TensorFlow - Huggingface - GitHub Actions - Docker}
% \end{justify}
% \vspace{\sectionpadding}
\cvsect{Interests}

\parbox[t]{\textwidth}{
    \begin{tabularx}{\textwidth}{>{\raggedright\arraybackslash}X >{\centering\arraybackslash}X >{\raggedleft\arraybackslash}X}
        triathlon & cycling & photography \\
    \end{tabularx}
    \vspace{1ex}
}


\end{document}
